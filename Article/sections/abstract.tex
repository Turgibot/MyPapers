% Abstract: add a concise summary of objectives, methods, results, and contributions.
Accurate Estimated Time of Arrival (ETA) prediction is crucial for modern navigation systems and intelligent transportation applications. 
While existing approaches rely on static routing algorithms or machine learning models that infer likely paths between origin and destination, they fail to leverage the user's actual intended route and the dynamic nature of traffic interactions. 
This paper introduces a novel Dynamic Graph Neural Network (DGNN) architecture that explicitly incorporates user route intent and temporal memory to achieve superior ETA prediction accuracy.

Our approach represents the transportation network as a dynamic multi-relational graph where junctions and vehicles are nodes, connected by typed edges capturing road segments, vehicle traversals, traffic interactions, and route intentions. The model employs a mixture-of-experts architecture with specialized experts and a memory layer that records historical snapshots to capture temporal dependencies. Unlike existing benchmarks that lack explicit route representation, our dataset unifies junction states, vehicle dynamics, and pre-planned route information, enabling the model to condition predictions on the actual path intended by travelers.

Experimental results on simulated traffic data demonstrate the effectiveness of our approach. The mixture-of-experts architecture with route awareness proves essential for handling heterogeneous traffic regimes and dynamic conditions.

\textbf{Key contributions:} (1) A dynamic graph-based formulation for ETA prediction incorporating explicit user route intent, (2) A DGNN architecture with mixture-of-experts and temporal memory, (3) A comprehensive evaluation highlighting the benefits of route awareness and dynamic graph modeling. 
