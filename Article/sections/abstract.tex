\begin{abstract}
    Accurate Estimated Time of Arrival (ETA) prediction is crucial for modern navigation systems and intelligent transportation applications. While existing approaches rely on static routing algorithms or machine learning models that infer likely paths between origin and destination, they fail to leverage the user's actual intended route and the dynamic nature of traffic interactions. This paper introduces a novel Dynamic Graph Neural Network (DGNN) architecture that explicitly incorporates user route intent and temporal memory to achieve superior ETA prediction accuracy.
    
    Our approach represents the transportation network as a dynamic multi-relational graph where junctions and vehicles are nodes, connected by typed edges capturing road segments, vehicle traversals, traffic interactions, and route intentions. The model employs a mixture-of-experts architecture with specialized experts and a memory layer that records historical snapshots to capture temporal dependencies. Unlike existing benchmarks that lack explicit route representation, our dataset unifies junction states, vehicle dynamics, and pre-planned route information, spanning \textasciitilde927,000 trips with durations ranging from 3 to 78 minutes and 80,640 graph snapshots collected over four simulated weeks.
    
    Experimental results on simulated traffic data demonstrate the effectiveness of our approach, achieving a best validation error of MAE = 24.6 seconds for short trips, 39.6 seconds for medium trips, and 80.2 seconds for long trips (overall MAE $\approx$ 47.8 seconds). Progressive ablation studies highlight the contribution of each modeling component: transitioning from a structural baseline to interaction-aware, demand-aware, and finally the full intent-aware model yields consistent improvements, with long trips remaining the most challenging. These findings confirm that explicit route representation, dynamic node modeling, and expressive edge features are the primary factors enabling accurate ETA prediction, while the mixture-of-experts framework further enhances robustness under heterogeneous traffic regimes and dynamic conditions.
    \end{abstract}
    