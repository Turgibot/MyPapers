\section{Introduction}
Estimated Time of Arrival (ETA) prediction is a fundamental component of modern navigation systems and intelligent transportation applications. Accurate ETA enables commuters to make informed travel decisions, supports fleet management and logistics operations, and reduces congestion by distributing demand across the road network. With the ubiquity of GPS-enabled devices and connected vehicles, navigation platforms such as Google Maps and Waze \cite{derrowpinion2021googlemaps,hoseinzadeh2020waze,amin-naseri2018waze} have transformed how travelers plan and adapt their journeys. Despite this progress, ETA estimation remains challenging due to the dynamic and stochastic nature of urban traffic.

Traditional approaches rely on shortest-path algorithms such as Dijkstra's algorithm \cite{dijkstra1959}, which efficiently compute routes under static conditions but fail to capture evolving congestion or vehicle interactions. As a result, travel times obtained from such methods can diverge substantially from reality when conditions change during a trip.

With the growth of urban mobility datasets, machine-learning models have been applied to ETA prediction. Tree-based methods such as XGBoost \cite{chen2016xgboost} leverage handcrafted features and perform well on aggregated trip records, including NYC Taxi and Porto Taxi \cite{nyc_tlc,moreira2013porto}. Deep learning approaches improve accuracy by modeling temporal patterns and spatio-temporal context along a route. For example, DeepTTE learns ETA from raw GPS traces \cite{wang2018deeptte}, TADNM incorporates transportation-mode awareness \cite{xu2020tadnm}, MetaTTE applies meta-learning for cross-city generalization \cite{wang2022metatte}, and STAD corrects routing-engine outputs using spatio-temporal adjustments \cite{abbar2020stad}.

In parallel, graph-based spatio-temporal learning has advanced traffic prediction by representing road networks as graphs and capturing dynamics through diffusion or convolution operators (e.g., DCRNN, ST-GCN, Graph WaveNet) \cite{li2018dcrnn,yu2018stgcn,wu2019graphwavenet}. However, existing benchmarks such as NYC, Porto, Chengdu/DiDi, and Geolife \cite{nyc_tlc,moreira2013porto,didi2016,zheng2012geolife} lack explicit representation of pre-planned routes. Models must therefore infer likely paths between origin and destination, introducing ambiguity and reducing accuracy. Prior work has proposed incorporating future-traffic–aware route selection to improve ETA estimation \cite{voloch2021}. We extend this direction by explicitly modeling pre-computed source–destination routes within a dynamic spatio-temporal graph, combined with a graph-based mixture-of-experts architecture.

The main contributions of this work are as follows:
\begin{itemize}
    \item \textbf{Dataset:} We construct a dynamic graph-based dataset that unifies junction states, vehicle dynamics, and explicit pre-planned route information. 
    \item \textbf{Model:} We propose a Dynamic Graph Neural Network (DGNN) architecture that integrates graph attention, temporal encoding, and mixture-of-experts specialization for accurate ETA prediction. 
    \item \textbf{Evaluation:} We conduct comprehensive experiments on simulated traffic data, including ablation studies, demonstrating that explicit route representation yields substantial improvements compared to route-agnostic variants. 
\end{itemize}
