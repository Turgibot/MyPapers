\section{Introduction}
Estimated Time of Arrival (ETA) prediction is a fundamental component of modern navigation systems and intelligent transportation applications. Accurate ETA enables commuters to make informed travel decisions, supports fleet management and logistics operations, and reduces congestion by distributing demand across the road network. With the ubiquity of GPS-enabled devices and connected vehicles, navigation platforms such as Google Maps and Waze \cite{derrowpinion2021googlemaps,hoseinzadeh2020waze,amin-naseri2018waze} have transformed how travelers plan and adapt their journeys. However, despite significant progress, ETA estimation remains challenging due to the dynamic and stochastic nature of urban traffic.

Traditional approaches to ETA computation rely on shortest-path algorithms such as Dijkstra's algorithm \cite{dijkstra1959}, which provide efficient solutions under static conditions but do not account for evolving congestion or interaction effects between vehicles. As a result, travel times computed via such methods can diverge substantially from reality when traffic conditions fluctuate during the trip.

With the growth of urban mobility datasets, machine-learning approaches have been introduced. Tree-based models such as XGBoost \cite{chen2016xgboost} predict ETA from handcrafted features and perform well on aggregated trip records like NYC Taxi and Porto Taxi \cite{nyc_tlc,moreira2013porto}. Deep learning methods improve performance by modeling temporal patterns and spatio-temporal context along a route: DeepTTE learns ETA from raw GPS traces \cite{wang2018deeptte}; TADNM incorporates transportation-mode awareness \cite{xu2020tadnm}; MetaTTE applies meta-learning for cross-city generalization \cite{wang2022metatte}; and STAD corrects routing-engine outputs using spatio-temporal adjustments \cite{abbar2020stad}.

In parallel, graph-based spatio-temporal learning has advanced traffic prediction by representing road networks as graphs and learning their dynamics with diffusion or convolution operators (e.g., DCRNN, ST-GCN, Graph WaveNet) \cite{li2018dcrnn,yu2018stgcn,wu2019graphwavenet}. Nevertheless, existing benchmarks (NYC, Porto, Chengdu/DiDi, Geolife) \cite{nyc_tlc,moreira2013porto,didi2016,zheng2012geolife} lack explicit representation of the user's pre-planned route. Models must therefore infer the likely path between origin and destination, introducing ambiguity and reducing prediction accuracy. Prior work proposed incorporating future-traffic–aware route selection to improve ETA accuracy \cite{voloch2021}. We build on this direction by explicitly modeling user intent and the pre-planned route within a dynamic spatio-temporal graph, and by learning ETA with a graph-based mixture-of-experts architecture.

Our contributions are: (i) a dynamic graph-based dataset unifying junction states, vehicle dynamics, and explicit pre-planned route information; (ii) a DGNN architecture that combines graph attention, temporal encoding, and mixture-of-experts specialization for accurate ETA prediction; and (iii) comprehensive experiments on simulated traffic data demonstrating significant gains over route-unaware baselines.
