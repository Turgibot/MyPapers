\section{Introduction}
Estimated Time of Arrival (ETA) prediction is a fundamental component of modern navigation systems and intelligent transportation applications. Accurate ETA enables commuters to make informed travel decisions, supports fleet management and logistics operations, and reduces congestion by distributing demand across the road network. With the ubiquity of GPS-enabled devices and connected vehicles, navigation platforms such as Google Maps and Waze \cite{derrowpinion2021googlemaps,hoseinzadeh2020waze,amin-naseri2018waze} have transformed how travelers plan and adapt their journeys. Despite this progress, ETA estimation remains challenging due to the dynamic and stochastic nature of urban traffic.

Traditional approaches rely on shortest-path algorithms such as Dijkstra's algorithm \cite{dijkstra1959}, which efficiently compute routes under static conditions but fail to capture evolving congestion or vehicle interactions. As a result, travel times obtained from such methods can diverge substantially from reality when conditions change during a trip.

With the growth of urban mobility datasets, machine-learning models have been applied to ETA prediction. Tree-based methods such as XGBoost \cite{chen2016xgboost} leverage handcrafted features and perform well on aggregated trip records, including NYC Taxi and Porto Taxi \cite{nyc_tlc,moreira2013porto}. Deep learning approaches improve accuracy by modeling temporal patterns and spatio-temporal context along a route. For example, DeepTTE learns ETA from raw GPS traces \cite{wang2018deeptte}, TADNM incorporates transportation-mode awareness \cite{xu2020tadnm}, MetaTTE applies meta-learning for cross-city generalization \cite{wang2022metatte}, and STAD corrects routing-engine outputs using spatio-temporal adjustments \cite{abbar2020stad}.

In parallel, graph-based spatio-temporal learning has advanced traffic prediction by representing road networks as graphs and capturing dynamics through diffusion or convolution operators (e.g., DCRNN, ST-GCN, Graph WaveNet) \cite{li2018dcrnn,yu2018stgcn,wu2019graphwavenet}. However, existing benchmarks such as NYC, Porto, Chengdu/DiDi, and Geolife \cite{nyc_tlc,moreira2013porto,didi2016,zheng2012geolife} lack explicit representation of pre-planned routes. Models must therefore infer likely paths between origin and destination, introducing ambiguity and reducing accuracy. We model traffic as a dynamic, multi-relational graph with static junction/road edges and dynamic vehicle/interaction edges (see Fig.~\ref{fig:dyn-graph}). Prior work has proposed incorporating future-traffic–aware route selection to improve ETA estimation \cite{voloch2021}. We extend this direction by explicitly encoding the remaining planned route (route intent) and learning temporal context from 
stable road edges across a short history, while using the full graph at the last snapshot to predict per-vehicle ETA.

\noindent\textbf{Summary of results.} On a four-week urban simulation, a simple average-time baseline attains 260.6\,s MAE; our route-aware non-temporal variant achieves 58.4\,s MAE, and the temporal route-aware model reaches 46.2\,s MAE. These gains highlight the value of making route intent explicit and of learning temporal signals from stable road infrastructure rather than noisy, transient vehicle interactions.

\noindent\textbf{Paper outline.} We introduce the dynamic graph representation and problem setup, summarize dataset characteristics and splits, detail the model architecture, and then present experiments and ablations followed by discussion and conclusions.

The main contributions of this work are as follows:
\begin{itemize}
    \item \textbf{Dataset/Representation:} We introduce a dynamic, multi-relational traffic graph that unifies static junction nodes and road edges with dynamic vehicle nodes and interaction edges, and provides explicit remaining-route annotations per vehicle to expose route intent. 
    \item \textbf{Modeling:} We design a route-aware spatio-temporal GNN that learns temporal context from stable road edges over the first $H{-}1$ snapshots and fuses it with full-graph features at prediction time, using a sparse Mixture-of-Experts head for specialization. 
    \item \textbf{Results/Analysis:} We show that MAE improves from 260.6\,s (average baseline) to 58.4\,s with route awareness and to 46.2\,s with temporal route awareness; ablations identify route intent and road-based temporal aggregation as the primary drivers of the gains. 
\end{itemize}
