\section{Methodology}

\subsection{Dynamic Graph Representation}
We represent the transportation system at snapshot time $t$ as a directed multi-relational graph
\[
G_t = \big(V_t, E_t, \{A_t^{(\rho)}, W_t^{(\rho)}\}_{\rho\in\mathcal{R}}\big),
\]
where:
\begin{itemize}
    \item $V_t = V^j \cup V_t^v$ is the set of nodes at time $t$, consisting of
        \begin{itemize}
            \item $V^j$: static junction nodes, representing intersections in the road network,
            \item $V_t^v$: dynamic vehicle nodes, representing vehicles present at time $t$.
        \end{itemize}
    \item $E_t$ is the set of directed edges at time $t$, partitioned by relation type $\rho$.
    \item $\mathcal{R}$ is the set of relation types (road, traversal, interaction).
    \item $A_t^{(\rho)} \in \{0,1\}^{|V_t|\times |V_t|}$ is the binary adjacency matrix for relation $\rho$ at time $t$.
    \item $W_t^{(\rho)} \in \mathbb{R}_{\ge 0}^{|V_t|\times |V_t|}$ is the optional weighted adjacency for relation $\rho$ (e.g., distance-based interaction weights).
\end{itemize}

We define three relation types:
\begin{enumerate}
    \item \textbf{Road edges ($\rho=\text{road}$):} For adjacent junctions $(u,v)$ with legal direction $u\!\to\!v$, we add $(u,v)\in E_t^{(\text{road})}$, encoding static road topology.
    \item \textbf{Traversal edges ($\rho=\text{trav}$):} If vehicle $v_i^t$ occupies segment $(a,b)$, we add $(a,v_i^t)$ and $(v_i^t,b)$, linking the vehicle to its upstream and downstream junctions.
    \item \textbf{Interaction edges ($\rho=\text{inter}$):} For vehicles $v_i^t,v_j^t$ on the same segment with spacing $d_{ij}(t)\le\varepsilon$ and aligned headings, we add $(v_i^t,v_j^t)$ weighted by $\omega(d_{ij}(t)) = e^{-d_{ij}(t)/\lambda}$.
\end{enumerate}

Each node and edge carries feature vectors capturing static attributes (e.g., number of lanes, road type), dynamic states (e.g., vehicle speed, occupancy), and local traffic interactions.

\paragraph{Route intent.}
In addition to graph-based relations, each vehicle node includes a feature called \texttt{vehicle\_route\_left}, which encodes the sequence of upcoming edges along its pre-computed source--destination path. 
This representation explicitly provides route intent, ensuring predictions are conditioned on the actual trajectory a vehicle will follow rather than an inferred path. 
The sequence is later processed by a Route Encoder, described below.

\subsection{Temporal Windowing}
ETA prediction requires reasoning over temporal dynamics. We construct a window of $H$ consecutive snapshots:
\[
\mathcal{G}_{t-H+1:t} = \{G_\tau\}_{\tau=t-H+1}^t,
\]
where $H=30$ in our experiments (30-second interval). The prediction target is the ETA of vehicles present in the final snapshot $G_t$.

\subsection{Model Architecture}
Our Dynamic Graph Neural Network (DGNN) integrates spatial, temporal, and route information. The main components are:

\begin{itemize}
    \item \textbf{Graph Encoder:} Each snapshot $G_t$ is processed by a multi-layer GATv2-based encoder with residual connections and edge features, producing embeddings for junction and vehicle nodes.
    \item \textbf{Temporal Aggregator:} The sequence of snapshot embeddings is summarized into fixed-size context vectors by concatenating the mean of junction embeddings and the mean of vehicle embeddings at each step. A GRU then processes the resulting sequence of $H$ context vectors to produce a history-aware temporal context. This vector is broadcast to all vehicles at $t^*$, allowing each prediction to incorporate recent traffic evolution.
    \item \textbf{Vehicle Selection:} From the final snapshot $G_t$, we retain only vehicle node embeddings, as these are the prediction targets for ETA.
    \item \textbf{Route Encoder:} Each vehicle’s remaining route (the \texttt{vehicle\_route\_left} feature) is embedded via an edge-ID embedding with mean pooling, then concatenated with its vehicle embedding.
    \item \textbf{Fusion and MoE Head:} Vehicle embeddings, enriched with route and temporal context, are fused through a feed-forward network and routed to a sparse Top-$k$ Mixture-of-Experts (MoE) head, where specialized experts capture heterogeneous traffic regimes. The output is the ETA prediction for each vehicle.
\end{itemize}

% ===== TikZ Diagram of Model Architecture =====
\begin{figure}[t]
    \centering
    \resizebox{0.95\columnwidth}{!}{%
    % TikZ diagram: Temporal MoE ETA model
% This file is included by sections/methodology.tex

% Styles used by nodes in this diagram
\tikzset{
  block/.style = {draw, rounded corners=2pt, thick, align=center, inner sep=6pt, fill=black!3},
  small/.style = {draw, rounded corners=2pt, align=center, inner sep=4pt, fill=black!3},
  op/.style    = {block, fill=blue!6},
  opt/.style   = {block, fill=green!7},
  moe/.style   = {block, fill=orange!12}
}

\begin{tikzpicture}[>=Latex, node distance=12mm]

% --- Snapshots as an overlapped deck (shared encoder) ---
\node[block, minimum width=40mm, minimum height=16mm] (snapdeck)
  {Snapshots $t{-}H{+}1,\ldots,t$ \\ \footnotesize Dynamic graphs ($x$, $edge\_index$, $edge\_attr$)};

% faint background cards behind the main one (deck effect)
\begin{scope}[on background layer]
  % soft shadow under/behind the top card (horizontal alignment)
  \node[fill=black!12, draw=none, rounded corners=2pt, minimum width=42mm, minimum height=18mm]
    at ($(snapdeck.center)+(1.2mm,0mm)$) {};
  % aligned background cards with thicker frames (exact same size as snapdeck)
  % use snapdeck's corners to draw identically sized rectangles behind it
  \draw[rounded corners=2pt, line width=1.4pt, draw=black!70, fill=white]
    ($(snapdeck.south west)+(6mm,6mm)$) rectangle ($(snapdeck.north east)+(6mm,6mm)$);
  \draw[rounded corners=2pt, line width=1.4pt, draw=black!70, fill=white]
    ($(snapdeck.south west)+(5mm,5mm)$) rectangle ($(snapdeck.north east)+(5mm,5mm)$);
  \draw[rounded corners=2pt, line width=1.4pt, draw=black!70, fill=white]
    ($(snapdeck.south west)+(4mm,4mm)$) rectangle ($(snapdeck.north east)+(4mm,4mm)$);
  \draw[rounded corners=2pt, line width=1.4pt, draw=black!70, fill=white]
    ($(snapdeck.south west)+(3mm,3mm)$) rectangle ($(snapdeck.north east)+(3mm,3mm)$);
  \draw[rounded corners=2pt, line width=1.4pt, draw=black!70, fill=white]
    ($(snapdeck.south west)+(2mm,2mm)$) rectangle ($(snapdeck.north east)+(2mm,2mm)$);
  \draw[rounded corners=2pt, line width=1.4pt, draw=black!70, fill=white]
    ($(snapdeck.south west)+(1mm,1mm)$)  rectangle ($(snapdeck.north east)+(1mm,1mm)$);
\end{scope}

% --- Single Graph Encoder (shared across time) ---
\node[op, below=of snapdeck, minimum width=45mm] (encoder)
  {Graph Encoder \\ \footnotesize (shared across snapshots)};

% --- Temporal module (predict-on-last in impl) ---
\node[op, below=15mm of encoder, minimum width=60mm] (temp)
  {Temporal module GRU \\ \footnotesize predict at last $t^*$ };

% --- Vehicle selection ---
\node[small, below=of temp] (vehsel) {Select vehicle nodes at $t^*$};

% --- Route encoder (optional, Full) ---
\node[opt, right=20mm of vehsel, align=left, minimum width=55mm] (routeenc)
  {Route Encoder\\ \footnotesize edge-id embedding + mean pooling\\
   \footnotesize inputs: $vehicle\_route\_left$, $vehicle\_route\_left\_splits$};

% --- Fusion / Router / Experts / Output ---
\node[op,   below=of vehsel,  minimum width=70mm] (fusion)  {Fusion MLP};
\node[moe,  below=of fusion,  minimum width=70mm] (router)  {Router (softmax, temperature, noise) \\ Top-$k$ selection};
\node[moe,  below=of router,  minimum width=70mm] (experts) {Experts (Residual MLPs) $\times E$, weighted by Top-$k$};
\node[block,below=of experts, minimum width=70mm] (pred)    {Per-vehicle ETA at $t^*$};

% --- Connections ---
\draw[->] (snapdeck) -- (encoder);
\draw[->] (encoder) -- (temp);
\draw[->] (temp) -- (vehsel);
\draw[->] (vehsel) -- (fusion);
\draw[->] (fusion) -- (router);
\draw[->] (router) -- (experts);
\draw[->] (experts) -- (pred);

% optional route path (Full)
\draw[->, dashed] (vehsel) -- (routeenc);
\draw[->, dashed] (routeenc.south) |- (fusion.east);

% small notes (optional)
% \node[above=0mm of snapdeck, font=\footnotesize] {Input from sliding window dataset};
% \node[right=2mm of pred,  font=\footnotesize, align=left] {target inverted to seconds for metrics};

\end{tikzpicture}
%
    }
    \caption{Temporal MoE ETA model. A window of $T{=}30$ dynamic graph snapshots (shown as an overlapped deck) is processed by a \emph{shared} Graph Encoder; prediction is made at the last snapshot $t^*$. In the Full variant, a Route Encoder summarizes each vehicle's remaining path before Fusion and a Top-$k$ MoE head produces per-vehicle ETA.}
    \label{fig:temporal-moe-eta}
\end{figure}
\subsection{Loss Functions and Training Protocol}
We train the model with supervised regression on ETA targets using mean absolute error (MAE) in seconds:
\[
\mathcal{L}_{\text{MAE}} = \frac{1}{N}\sum_{i=1}^N \big| \hat{y}_i - y_i \big|,
\]
where $\hat{y}_i$ and $y_i$ are the predicted and true ETAs. Additional evaluation metrics include RMSE, weighted absolute percentage error (WAPE), and percentile errors (P50, P90, P95). 
All experiments are run on the same dataset splits, enabling consistent comparisons.


\subsection{Loss Functions and Training Protocol}
We train the model with supervised regression on ETA targets using mean absolute error (MAE) in seconds:
\[
\mathcal{L}_{\text{MAE}} = \frac{1}{N}\sum_{i=1}^N \big| \hat{y}_i - y_i \big|,
\]
where $\hat{y}_i$ and $y_i$ are the predicted and true ETAs. Additional evaluation metrics include RMSE, weighted absolute percentage error (WAPE), and percentile errors (P50, P90, P95). 

The dataset spans four simulated weeks of traffic. We partition this chronologically into two weeks for training, one week for validation, and one week for testing. This split ensures that models are evaluated on non-overlapping time intervals that include different traffic patterns (rush hours, weekends, and long trips).

\paragraph{Ablation Variants.}
To quantify the contribution of each modeling component, we design several controlled ablations:
\begin{itemize}
    \item \textbf{None (structural baseline):} Road edges only; static features.
    \item \textbf{Weak (interaction-aware):} Adds dynamic traversal and interaction edges.
    \item \textbf{Medium (demand-aware):} Adds aggregate route-related features (e.g., edge route counts).
    \item \textbf{Full (intent-aware):} Adds explicit route encoder over each vehicle's remaining path (\texttt{vehicle\_route\_left}).
    \item \textbf{Full+Temporal:} Extends Full by concatenating a GRU-based temporal context aggregated from past snapshots.
    \item \textbf{Full+Memory:} Further extends Full+Temporal with a memory mechanism that accumulates historical contexts across windows, capturing longer-range dependencies beyond the local horizon.
\end{itemize}
All variants share the same training protocol and data splits, enabling a controlled comparison of structural, dynamic, demand, intent, and temporal features.
