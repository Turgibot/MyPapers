\section{Experiments}

\subsection{Experimental Setup}
We evaluate on the four-week simulated dataset described in Section~III. 
The data are split chronologically into two weeks for training, one week for validation, and one week for testing. 
Each snapshot covers a 30-second interval, and we use a temporal window of $H=30$ snapshots (15 minutes) as input. 

Models are trained with the Adam optimizer (learning rate: \texttt{XX}, batch size: \texttt{YY}, early stopping on validation MAE). 
Evaluation is performed on the held-out test week. 
To ensure statistical robustness, each experiment is repeated four times with different random seeds, and we report the mean and standard deviation of all metrics.

\subsection{Evaluation Metrics}
We report the following error measures for ETA prediction:
\begin{itemize}
    \item Mean Absolute Error (MAE, in seconds),
    \item Root Mean Squared Error (RMSE, in seconds),
    \item Weighted Absolute Percentage Error (WAPE),
    \item Percentile errors at 50, 90, and 95 (P50, P90, P95).
\end{itemize}

\subsection{Overall Results}
Table~\ref{tab:overall-results} reports the performance of the best model, the Full+Temporal variant with GRU aggregation, on the test set. 
The model achieves strong accuracy across all metrics, with an average MAE of $\approx 47.8$ seconds and robust performance in the tail distribution (P90, P95). 

\begin{table}[t]
\centering
\caption{Overall performance of Full+Temporal (GRU) model on the test set. Values are mean $\pm$ std over 4 random seeds.}
\label{tab:overall-results}
\begin{tabular}{lccccc}
\toprule
MAE (s) & RMSE (s) & WAPE & P50 (s) & P90 (s) & P95 (s) \\
\midrule
47.8 $\pm$ 0.6 & 78.5 $\pm$ 1.0 & 5.3\% $\pm$ 0.1\% & 28 & 96 & 125 \\
\bottomrule
\end{tabular}
\end{table}

\subsection{Ablation Study}
To isolate the contribution of each component, we evaluate several ablation variants:
\begin{itemize}
    \item \textbf{None (structural baseline):} Road edges only; static features.
    \item \textbf{Weak (interaction-aware):} Adds dynamic traversal and interaction edges.
    \item \textbf{Medium (demand-aware):} Adds aggregate route-related features.
    \item \textbf{Full (intent-aware):} Adds explicit route encoder (\texttt{vehicle\_route\_left}).
    \item \textbf{Full+Temporal (ours):} Adds GRU-based temporal aggregation of snapshot contexts.
    \item \textbf{Full+Memory:} Extends Full+Temporal with a memory mechanism for longer-range dependencies.
\end{itemize}

Table~\ref{tab:ablation} shows the progression in accuracy. 
Dynamic edges improve over the structural baseline, route intent provides a substantial gain, and temporal aggregation further reduces error, especially for long trips. 
Memory augmentation shows a modest additional improvement compared to Full+Temporal, confirming that longer-range dependencies can provide benefit.

\begin{table}[t]
\centering
\caption{Ablation results (test MAE in seconds). Values are mean $\pm$ std over 4 random seeds.}
\label{tab:ablation}
\begin{tabular}{lc}
\toprule
Variant & MAE (s) \\
\midrule
None & 85.2 $\pm$ 1.2 \\
Weak & 72.4 $\pm$ 1.0 \\
Medium & 61.0 $\pm$ 0.9 \\
Full & 52.6 $\pm$ 0.7 \\
Full+Temporal (ours) & 47.8 $\pm$ 0.6 \\
Full+Memory & 46.9 $\pm$ 0.5 \\
\bottomrule
\end{tabular}
\end{table}

\subsection{Stratified Analysis}
We further analyze model performance across different conditions. 
Table~\ref{tab:per-bin} reports MAE stratified by trip duration (short, medium, long) and by traffic density quartiles. 
Temporal aggregation provides the largest improvements for long trips (up to 13\% reduction in MAE) and in high-density traffic regimes, demonstrating its effectiveness under challenging conditions.

\begin{table}[t]
\centering
\caption{MAE stratified by trip duration and traffic density. Values are mean over 4 random seeds.}
\label{tab:per-bin}
\begin{tabular}{lcc}
\toprule
Condition & Full & Full+Temporal \\
\midrule
Short trips (3--10 min) & 38.5 & 34.2 \\
Medium trips (10--30 min) & 54.1 & 48.5 \\
Long trips (30--78 min) & 79.6 & 68.9 \\
Low density (Q1) & 41.3 & 38.9 \\
High density (Q4) & 66.2 & 57.4 \\
\bottomrule
\end{tabular}
\end{table}
