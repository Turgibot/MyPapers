\section{Related Work}

\subsection{Classical pathfinding and routing}
On static graphs \(G=(V,E)\) with non-negative edge costs, shortest paths are classically computed via Dijkstra's algorithm \cite{dijkstra1959}. Alternatives such as Bellman--Ford \cite{bellman1958routing} and bidirectional search \cite{pohl1971bi} offer different trade-offs between generality and efficiency. To enable web-scale applications, numerous speed-up techniques have been developed. These include A*/ALT landmarks \cite{goldberg2005alt}, which guide search with precomputed heuristics; Contraction Hierarchies \cite{geisberger2008ch}, which accelerate queries by shortcutting less important edges; highway-dimension theory \cite{abraham2010highwaydimension}, which formalizes why sparse “highway” structures allow fast queries; and Customizable Route Planning \cite{delling2011crp}, which separates preprocessing from query time to support rapid updates. Extensions such as multi-criteria routing and $k$-shortest path algorithms \cite{yen1971ksp} broaden the option set by considering cost trade-offs or alternatives. While these methods are efficient, travel-time estimates are typically obtained from static or exogenous edge-time models. As a result, they cannot account for real-time congestion, stochastic traffic dynamics, or interactions between vehicles, limiting their predictive accuracy.

\subsection{Learning-based ETA and traffic forecasting}
The proliferation of large-scale urban mobility datasets has motivated learning-based approaches to ETA prediction. Classical machine-learning models such as gradient boosting \cite{chen2016xgboost} leverage handcrafted features (e.g., trip distance, time-of-day, origin-destination statistics) and have shown strong performance on aggregated trip datasets, notably NYC Taxi and Porto Taxi \cite{nyc_tlc,moreira2013porto}. These methods are simple and interpretable, but their reliance on aggregate features prevents them from capturing fine-grained spatio-temporal dependencies.

Deep learning approaches address this gap by directly modeling temporal patterns and route context. DeepTTE learns ETA from raw GPS traces with recurrent and convolutional layers \cite{wang2018deeptte}. TADNM incorporates transportation-mode awareness to adapt predictions across different mobility contexts \cite{xu2020tadnm}. MetaTTE applies meta-learning techniques for cross-city generalization, allowing a model trained in one region to adapt quickly to another \cite{wang2022metatte}. STAD refines routing-engine outputs by applying spatio-temporal corrections to the estimated times \cite{abbar2020stad}. These models demonstrate that neural networks can effectively capture sequential and contextual features beyond static trip attributes.

In parallel, spatio-temporal graph neural networks (STGNNs) have advanced traffic prediction by explicitly representing road networks as graphs. Diffusion Convolutional Recurrent Neural Networks (DCRNN) \cite{li2018dcrnn}, Spatio-Temporal Graph Convolutional Networks (ST-GCN) \cite{yu2018stgcn}, and Graph WaveNet \cite{wu2019graphwavenet} propagate information across road graphs to capture local correlations and temporal evolution. These models achieve state-of-the-art results in short-term traffic forecasting, such as predicting traffic flow or speed on sensor-equipped road segments. However, widely used benchmarks (NYC Taxi, Porto Taxi, Chengdu/DiDi, Geolife) \cite{nyc_tlc,moreira2013porto,didi2016,zheng2012geolife} lack explicit representation of pre-planned routes. Models trained on these datasets must infer the likely path between an origin and destination, which introduces ambiguity and reduces ETA prediction accuracy. Moreover, benchmark datasets are often limited to aggregated trip records or fixed sensor locations, restricting their ability to model vehicle-level interactions.

\subsection{Intent- and route-aware prediction}
Accurately modeling route intent is critical for ETA prediction, as travel time depends strongly on the chosen path. Prior work has attempted to address this in several ways. Voloch et al. \cite{voloch2021} incorporate future-traffic--aware route selection, showing that improved routing decisions can reduce ETA errors. In the trajectory-prediction literature, destination-aware models leverage partial trajectories and contextual cues to infer intent, yielding better predictions of future positions. Similarly, intent modeling has been applied in ride-hailing and fleet management scenarios, where anticipated destinations influence dispatch and rebalancing strategies. These approaches demonstrate that explicitly reasoning about intent can improve both accuracy and downstream decision-making.

Nevertheless, most intent-aware works either focus on trajectory forecasting or optimize route selection strategies, rather than integrating intent into the ETA prediction model itself. Furthermore, they typically rely on datasets that lack explicit representations of the complete pre-planned route. In contrast, our work directly integrates pre-computed source--destination routes into a dynamic spatio-temporal graph framework. By representing both junctions and vehicles as nodes, and including typed edges for road segments, vehicle interactions, and route associations, we unify structural, dynamic, and intent-related features in a single model. This design bridges the gap between routing, trajectory forecasting, and ETA prediction, enabling fine-grained spatio-temporal learning conditioned on actual paths rather than inferred ones.
