% Options for packages loaded elsewhere
\PassOptionsToPackage{unicode}{hyperref}
\PassOptionsToPackage{hyphens}{url}
\PassOptionsToPackage{dvipsnames,svgnames,x11names}{xcolor}
%
\documentclass[
  10pt,
  letterpaper,
  lettersize,
  journal]{IEEEtran}

\usepackage{amsmath,amssymb}
\usepackage{iftex}
\ifPDFTeX
  \usepackage[T1]{fontenc}
  \usepackage[utf8]{inputenc}
  \usepackage{textcomp} % provide euro and other symbols
\else % if luatex or xetex
  \usepackage{unicode-math}
  \defaultfontfeatures{Scale=MatchLowercase}
  \defaultfontfeatures[\rmfamily]{Ligatures=TeX,Scale=1}
\fi
\usepackage{lmodern}
\ifPDFTeX\else  
    % xetex/luatex font selection
\fi
% Use upquote if available, for straight quotes in verbatim environments
\IfFileExists{upquote.sty}{\usepackage{upquote}}{}
\IfFileExists{microtype.sty}{% use microtype if available
  \usepackage[]{microtype}
  \UseMicrotypeSet[protrusion]{basicmath} % disable protrusion for tt fonts
}{}
\makeatletter
\@ifundefined{KOMAClassName}{% if non-KOMA class
  \IfFileExists{parskip.sty}{%
    \usepackage{parskip}
  }{% else
    \setlength{\parindent}{0pt}
    \setlength{\parskip}{6pt plus 2pt minus 1pt}}
}{% if KOMA class
  \KOMAoptions{parskip=half}}
\makeatother
\usepackage{xcolor}
\usepackage[margin=1in]{geometry}
\setlength{\emergencystretch}{3em} % prevent overfull lines
\setcounter{secnumdepth}{5}
% Make \paragraph and \subparagraph free-standing
\makeatletter
\ifx\paragraph\undefined\else
  \let\oldparagraph\paragraph
  \renewcommand{\paragraph}{
    \@ifstar
      \xxxParagraphStar
      \xxxParagraphNoStar
  }
  \newcommand{\xxxParagraphStar}[1]{\oldparagraph*{#1}\mbox{}}
  \newcommand{\xxxParagraphNoStar}[1]{\oldparagraph{#1}\mbox{}}
\fi
\ifx\subparagraph\undefined\else
  \let\oldsubparagraph\subparagraph
  \renewcommand{\subparagraph}{
    \@ifstar
      \xxxSubParagraphStar
      \xxxSubParagraphNoStar
  }
  \newcommand{\xxxSubParagraphStar}[1]{\oldsubparagraph*{#1}\mbox{}}
  \newcommand{\xxxSubParagraphNoStar}[1]{\oldsubparagraph{#1}\mbox{}}
\fi
\makeatother


\providecommand{\tightlist}{%
  \setlength{\itemsep}{0pt}\setlength{\parskip}{0pt}}\usepackage{longtable,booktabs,array}
\usepackage{calc} % for calculating minipage widths
% Correct order of tables after \paragraph or \subparagraph
\usepackage{etoolbox}
\makeatletter
\patchcmd\longtable{\par}{\if@noskipsec\mbox{}\fi\par}{}{}
\makeatother
% Allow footnotes in longtable head/foot
\IfFileExists{footnotehyper.sty}{\usepackage{footnotehyper}}{\usepackage{footnote}}
\makesavenoteenv{longtable}
\usepackage{graphicx}
\makeatletter
\newsavebox\pandoc@box
\newcommand*\pandocbounded[1]{% scales image to fit in text height/width
  \sbox\pandoc@box{#1}%
  \Gscale@div\@tempa{\textheight}{\dimexpr\ht\pandoc@box+\dp\pandoc@box\relax}%
  \Gscale@div\@tempb{\linewidth}{\wd\pandoc@box}%
  \ifdim\@tempb\p@<\@tempa\p@\let\@tempa\@tempb\fi% select the smaller of both
  \ifdim\@tempa\p@<\p@\scalebox{\@tempa}{\usebox\pandoc@box}%
  \else\usebox{\pandoc@box}%
  \fi%
}
% Set default figure placement to htbp
\def\fps@figure{htbp}
\makeatother
% definitions for citeproc citations
\NewDocumentCommand\citeproctext{}{}
\NewDocumentCommand\citeproc{mm}{%
  \begingroup\def\citeproctext{#2}\cite{#1}\endgroup}
\makeatletter
 % allow citations to break across lines
 \let\@cite@ofmt\@firstofone
 % avoid brackets around text for \cite:
 \def\@biblabel#1{}
 \def\@cite#1#2{{#1\if@tempswa , #2\fi}}
\makeatother
\newlength{\cslhangindent}
\setlength{\cslhangindent}{1.5em}
\newlength{\csllabelwidth}
\setlength{\csllabelwidth}{3em}
\newenvironment{CSLReferences}[2] % #1 hanging-indent, #2 entry-spacing
 {\begin{list}{}{%
  \setlength{\itemindent}{0pt}
  \setlength{\leftmargin}{0pt}
  \setlength{\parsep}{0pt}
  % turn on hanging indent if param 1 is 1
  \ifodd #1
   \setlength{\leftmargin}{\cslhangindent}
   \setlength{\itemindent}{-1\cslhangindent}
  \fi
  % set entry spacing
  \setlength{\itemsep}{#2\baselineskip}}}
 {\end{list}}
\usepackage{calc}
\newcommand{\CSLBlock}[1]{\hfill\break\parbox[t]{\linewidth}{\strut\ignorespaces#1\strut}}
\newcommand{\CSLLeftMargin}[1]{\parbox[t]{\csllabelwidth}{\strut#1\strut}}
\newcommand{\CSLRightInline}[1]{\parbox[t]{\linewidth - \csllabelwidth}{\strut#1\strut}}
\newcommand{\CSLIndent}[1]{\hspace{\cslhangindent}#1}

\usepackage{amsmath,amsfonts}
\usepackage{algorithmic}
\usepackage{algorithm}
\usepackage{array}
\usepackage[caption=false,font=normalsize,labelfont=sf,textfont=sf]{subfig}
\usepackage{textcomp}
\usepackage{stfloats}
\usepackage{url}
\usepackage{verbatim}
\usepackage{graphicx}
\usepackage{cite}
\usepackage{balance}
\usepackage{amssymb}
\usepackage{amsfonts}

% IEEE-specific commands
\hyphenation{op-tical net-works semi-conduc-tor IEEE-Xplore}

% Running heads
\markboth{IEEE Transactions on Intelligent Transportation Systems,~Vol.~XX, No.~X, Month~2025}%
{Tordjman \MakeLowercase{\textit{et al.}}: Incorporating User Intent into Dynamic Spatio-Temporal Graph Neural Networks for ETA Prediction}

% Copyright line (will be added by IEEE)
\IEEEpubid{0000--0000/00\$00.00~\copyright~2025 IEEE}

% Override Quarto's automatic formatting to match IEEE template
\makeatletter
% Remove Quarto's automatic title formatting
\renewcommand{\maketitle}{%
  \newpage
  \null
  \vskip 2em%
  \begin{center}%
  \let \footnote \thanks
    {\LARGE \@title \par}%
    \vskip 1.5em%
    {\large
      \lineskip .5em%
      \begin{tabular}[t]{c}%
        \@author
      \end{tabular}\par}%
    \vskip 1em%
    {\large \@date}%
  \end{center}%
  \par
  \vskip 1.5em}
\makeatother

% Override abstract environment to use IEEE format
\renewenvironment{abstract}{%
  \small
  \begin{center}%
    {\bfseries \abstractname\vspace{-.5em}\vspace{0pt}}%
  \end{center}%
  \quotation
}{%
  \endquotation
}

% IEEE keywords are already defined in IEEEtran class
\makeatletter
\@ifpackageloaded{caption}{}{\usepackage{caption}}
\AtBeginDocument{%
\ifdefined\contentsname
  \renewcommand*\contentsname{Table of contents}
\else
  \newcommand\contentsname{Table of contents}
\fi
\ifdefined\listfigurename
  \renewcommand*\listfigurename{List of Figures}
\else
  \newcommand\listfigurename{List of Figures}
\fi
\ifdefined\listtablename
  \renewcommand*\listtablename{List of Tables}
\else
  \newcommand\listtablename{List of Tables}
\fi
\ifdefined\figurename
  \renewcommand*\figurename{Figure}
\else
  \newcommand\figurename{Figure}
\fi
\ifdefined\tablename
  \renewcommand*\tablename{Table}
\else
  \newcommand\tablename{Table}
\fi
}
\@ifpackageloaded{float}{}{\usepackage{float}}
\floatstyle{ruled}
\@ifundefined{c@chapter}{\newfloat{codelisting}{h}{lop}}{\newfloat{codelisting}{h}{lop}[chapter]}
\floatname{codelisting}{Listing}
\newcommand*\listoflistings{\listof{codelisting}{List of Listings}}
\makeatother
\makeatletter
\makeatother
\makeatletter
\@ifpackageloaded{caption}{}{\usepackage{caption}}
\@ifpackageloaded{subcaption}{}{\usepackage{subcaption}}
\makeatother

\usepackage{bookmark}

\IfFileExists{xurl.sty}{\usepackage{xurl}}{} % add URL line breaks if available
\urlstyle{same} % disable monospaced font for URLs
\hypersetup{
  pdftitle={Dynamic Graph Neural Networks for Accurate ETA prediction},
  pdfauthor={Guy Tordjman; Nadav Voloch},
  pdfkeywords={ETA prediction, graph neural networks, spatio-temporal
graphs, user intent, traffic forecasting},
  colorlinks=true,
  linkcolor={blue},
  filecolor={Maroon},
  citecolor={Blue},
  urlcolor={Blue},
  pdfcreator={LaTeX via pandoc}}


\title{Dynamic Graph Neural Networks for Accurate ETA prediction}
\author{Guy Tordjman \and Nadav Voloch}
\date{}

\begin{document}
\maketitle


\section{Abstract}\label{abstract}

Accurate Estimated Time of Arrival (ETA) prediction is crucial for
modern navigation systems and intelligent transportation applications.
While existing approaches rely on static routing algorithms or machine
learning models that infer likely paths between origin and destination,
they fail to leverage the user's actual intended route and the dynamic
nature of traffic interactions. This paper introduces a novel Dynamic
Graph Neural Network (DGNN) architecture that explicitly incorporates
user route intent and temporal memory to achieve superior ETA prediction
accuracy.

Our approach represents the transportation network as a dynamic
multi-relational graph where junctions and vehicles are nodes, connected
by typed edges capturing road segments, vehicle traversals, traffic
interactions, and route intentions. The model employs a
mixture-of-experts architecture with 6 specialized experts and a memory
layer that records historical snapshots to capture temporal
dependencies. Unlike existing benchmarks that lack explicit route
representation, our dataset unifies junction states, vehicle dynamics,
and pre-planned route information, enabling the model to condition
predictions on the actual path intended by travelers.

Experimental results on simulated traffic data demonstrate the
effectiveness of our approach. The model achieves a validation MAE of
{[}RESULTS\_PLACEHOLDER{]} seconds, representing a
{[}IMPROVEMENT\_PLACEHOLDER{]}\% improvement over baseline methods.
Performance analysis across different trip duration bins shows
particularly strong results for {[}DETAILED\_RESULTS\_PLACEHOLDER{]}.
The mixture-of-experts architecture with route awareness proves
essential for handling heterogeneous traffic regimes, with the router
entropy stabilizing at {[}ENTROPY\_PLACEHOLDER{]} indicating effective
expert specialization.

\textbf{Key contributions:} (1) First dynamic graph-based dataset for
ETA prediction incorporating explicit user route intent, (2) Novel DGNN
architecture with mixture-of-experts and temporal memory layers, (3)
Comprehensive evaluation demonstrating significant accuracy improvements
over existing approaches that lack route awareness and dynamic graph
modeling.

\section{Introduction}\label{introduction}

\IEEEPARstart{E}{stimated} Time of Arrival (ETA) prediction is a
fundamental component of modern navigation systems and intelligent
transportation applications. Accurate ETA enables commuters to make
informed travel decisions, supports fleet management and logistics
operations, and reduces congestion by distributing demand across the
road network. With the ubiquity of GPS-enabled devices and connected
vehicles, navigation platforms such as Google Maps and Waze {[}1{]},
{[}2{]}, {[}3{]} have transformed how travelers plan and adapt their
journeys. However, despite significant progress, ETA estimation remains
challenging due to the dynamic and stochastic nature of urban traffic.

Traditional approaches to ETA computation rely on shortest-path
algorithms such as Dijkstra's algorithm {[}4{]}, which provide efficient
solutions under static conditions but do not account for evolving
congestion or interaction effects between vehicles. As a result, travel
times computed via such methods can diverge substantially from reality
when traffic conditions fluctuate during the trip.

With the growth of urban mobility datasets, machine-learning approaches
have been introduced. Tree-based models such as XGBoost {[}5{]} predict
ETA from handcrafted features (e.g., distance, departure time,
day-of-week) and perform well on aggregated trip records like NYC Taxi
{[}6{]} and Porto Taxi {[}7{]}. Deep learning methods improve
performance by modeling temporal patterns and spatio-temporal context
along a route: DeepTTE learns ETA from raw GPS traces {[}8{]}; TADNM
incorporates transportation-mode awareness {[}9{]}; MetaTTE applies
meta-learning for cross-city generalization {[}10{]}; and STAD corrects
routing-engine outputs using spatio-temporal adjustments {[}11{]}.

In parallel, graph-based spatio-temporal learning has advanced traffic
prediction by representing road networks as graphs and learning their
dynamics with diffusion or convolution operators (e.g., DCRNN, ST-GCN,
Graph WaveNet) {[}12{]}, {[}13{]}, {[}14{]}. Nevertheless, existing
benchmarks (NYC, Porto, Chengdu/DiDi, Geolife) {[}6{]}, {[}7{]},
{[}15{]}, {[}16{]} lack explicit representation of the user's
\textbf{pre-planned route}. Models must therefore infer the likely path
between origin and destination, introducing ambiguity and reducing
prediction accuracy. Early work proposed incorporating
future-traffic--aware route selection to improve ETA accuracy {[}17{]}.
We build on this direction by explicitly modeling user intent and the
pre-planned route within a dynamic spatio-temporal graph, and by
learning ETA with a graph-based mixture-of-experts architecture.

This paper addresses these limitations by introducing a novel Dynamic
Graph Neural Network (DGNN) that explicitly incorporates user route
intent and temporal memory. Our approach represents the transportation
network as a dynamic multi-relational graph where both junctions and
vehicles are nodes, connected by typed edges that capture road segments,
vehicle traversals, traffic interactions, and route intentions. The
model employs a mixture-of-experts architecture with specialized experts
and a memory layer that records historical snapshots to capture temporal
dependencies.

\textbf{Contributions.}\\
1) We introduce the first dynamic graph-based dataset for ETA prediction
that unifies junction states, vehicle dynamics, and explicit pre-planned
route information.\\
2) We propose a novel DGNN architecture that combines graph attention,
temporal encoding, and mixture-of-experts specialization for accurate
ETA prediction.\\
3) We conduct comprehensive experiments on simulated traffic data,
demonstrating significant accuracy improvements over existing approaches
that lack route awareness and dynamic graph modeling.

\section{Related Work}\label{related-work}

\subsection{Classical pathfinding and
routing}\label{classical-pathfinding-and-routing}

Modern navigation stacks historically separate \textbf{route
computation} from \textbf{travel-time estimation}. On a static graph
\(G=(V,E)\) with non-negative edge costs, the canonical approach is to
compute the minimum-cost path using Dijkstra's label-setting algorithm
{[}4{]}, then obtain an ETA by summing edge weights along the selected
path. This paradigm assumes costs are fixed during the trip and ignores
how congestion evolves while the vehicle is en route.

\subsubsection{Static vs.~time-dependent
routing}\label{static-vs.-time-dependent-routing}

In practice, edge costs are \textbf{time-dependent}: the travel time of
a road segment \(e\in E\) is a function \(c_e(t)\) of the departure time
\(t\). The \textbf{time-dependent shortest path (TDSP)} problem
generalizes classical routing by optimizing arrival time given these
functions. Under the FIFO (first-in--first-out) property---i.e.,
departing later cannot lead to an earlier arrival---label-setting
methods can be adapted to TD networks; otherwise, label-correcting
techniques are required (with weaker optimality guarantees). Despite
this, most production systems still approximate TDSP by (i) computing a
route under ``current'' costs and (ii) aggregating ETAs along that
route, or by using coarse predictive layers on top of a static
route---both approaches remain \textbf{route-first} rather than
\textbf{ETA-first} and can drift as traffic evolves.

\subsubsection{Speed-up techniques for web-scale
routing}\label{speed-up-techniques-for-web-scale-routing}

Because city- and nation-scale graphs contain millions of nodes, raw
Dijkstra is too slow for interactive use. A large body of work speeds up
queries via preprocessing:

\begin{itemize}
\tightlist
\item
  \textbf{A*} with \textbf{ALT landmarks} adds an admissible heuristic
  via triangle inequalities to reduce the search space {[}18{]}.\\
\item
  \textbf{Contraction Hierarchies (CH)} contract low-importance vertices
  to build shortcuts, enabling millisecond-level exact queries on road
  networks {[}19{]}.\\
\item
  \textbf{Highway-dimension theory} explains why such hierarchies are
  effective on real networks and yields provable bounds {[}20{]}.\\
\item
  \textbf{Arc-flags/SHARC} and \textbf{Multi-Level/Customizable Route
  Planning (CRP)} partition the graph; CRP separates metric-independent
  preprocessing from a fast ``customization'' step for new cost profiles
  {[}21{]}.
\end{itemize}

These techniques deliver fast \textbf{static} (or piecewise-static)
queries. Extensions exist for \textbf{time-dependent} costs, but
accuracy hinges on the fidelity of \(c_e(t)\), and updates must be
frequent to track live congestion.

\subsubsection{Multi-criteria and k-shortest
paths}\label{multi-criteria-and-k-shortest-paths}

In urban routing, users often optimize multiple criteria (time, tolls,
reliability). \textbf{k-shortest paths} (e.g., Yen's algorithm)
enumerate alternatives for robustness or post-processing {[}22{]}, yet
the ETA still depends on exogenous edge-time models rather than
endogenous traffic interactions.

\subsubsection{Limitations for ETA
prediction}\label{limitations-for-eta-prediction}

Classical routing \textbf{treats travel time as input}, not as a learned
outcome of interacting flows. Even TDSP with high-quality \(c_e(t)\)
remains \textbf{myopic} if those functions are derived from
current/historical averages: it does not anticipate \textbf{how vehicles
currently en route will reshape future speeds} on the chosen path.
Consequently, route-first stacks can systematically under- or
over-estimate arrival times when congestion is building or dissipating.
This motivates \textbf{learning-based, graph-temporal} approaches that
(i) reason over interactions between vehicles and junctions, and (ii)
condition ETA on \textbf{the intended route} (when available) to remove
path ambiguity---directions pursued by our method.

\section{Methodology}\label{methodology}

\subsection{Graph representation and construction
(formal)}\label{graph-representation-and-construction-formal}

Let \(t\in\mathbb{Z}\) index uniformly spaced snapshots (interval
\(\Delta t\)). The transportation state at time \(t\) is a
\textbf{directed, multi-relational graph}
\[G_t=\big(V_t,\ E_t,\ \{A_t^{(\rho)}\}_{\rho\in\mathcal{R}}\big),\]
with \[V_t=V^{j}\ \cup\ V_t^{v},\qquad V^{j}\cap V_t^{v}=\varnothing,\]
where \(V^{j}\) are \textbf{junction nodes} (static across time) and
\(V_t^{v}\) are \textbf{vehicle nodes} present at time \(t\). Each
vehicle \(i\) maintains a persistent identity across time; we denote its
node at time \(t\) by \(v_i^{t}\in V_t^{v}\).

We consider a finite set of \textbf{edge relations} (types)
\[\mathcal{R}=\{\text{road},\ \text{trav},\ \text{inter},\ \text{intent}\}.\]
For each relation \(\rho\in\mathcal{R}\) we define a binary adjacency
\(A_t^{(\rho)}\in\{0,1\}^{|V_t|\times |V_t|}\) and an optional weight
matrix \(W_t^{(\rho)}\in\mathbb{R}_{\ge 0}^{|V_t|\times |V_t|}\). The
(typed) edge set is
\[E_t^{(\rho)}=\{(u,v): A_t^{(\rho)}[u,v]=1\},\qquad E_t=\biguplus_{\rho\in\mathcal{R}}E_t^{(\rho)}.\]

\textbf{Typed edges.} At time \(t\),

\begin{enumerate}
\def\labelenumi{\arabic{enumi})}
\item
  \textbf{Road segment edges} (\(\rho=\text{road}\)): for adjacent
  junctions \(u,v\in V^{j}\) with legal driving direction \(u\!\to\!v\),
  \[(u,v)\in E^{(\text{road})}_t,\quad A_t^{(\text{road})}[u,v]=1.\]
\item
  \textbf{Traversal edges} (\(\rho=\text{trav}\)): if vehicle
  \(v_i^{t}\) occupies directed segment \((a,b)\) with \(a,b\in V^{j}\),
  \[(a, v_i^{t})\in E_t^{(\text{trav})},\qquad (v_i^{t}, b)\in E_t^{(\text{trav})}.\]
\item
  \textbf{Interaction edges} (\(\rho=\text{inter}\)): for vehicles
  \(v_i^{t}, v_{j}^{t}\in V_t^{v}\) on the same segment (or lane group)
  with longitudinal spacing \(d_{ij}(t)\le \varepsilon\) and aligned
  headings,
  \[(v_i^{t}, v_{j}^{t})\in E_t^{(\text{inter})},\quad W_t^{(\text{inter})}[v_i^{t},v_{j}^{t}]=\omega\!\big(d_{ij}(t)\big),\]
  where \(\omega:\mathbb{R}_{\ge 0}\!\to\!\mathbb{R}_{\ge 0}\) is a
  monotone kernel (e.g., \(\omega(d)=\exp(-d/\lambda)\)).
\item
  \textbf{Intent edges} (\(\rho=\text{intent}\)): let the
  \textbf{planned route} for vehicle \(i\) at departure time
  \(t_0\le t\) be the ordered edge sequence
  \[\mathcal{R}_{i}=\big(e_1,e_2,\ldots,e_{K}\big),\qquad e_k=(u_k,u_{k+1})\in E^{(\text{road})}_{t_0}.\]
  We connect the vehicle to upcoming junctions along its route,
  \[(v_i^{t},\ u_{t,i}^{(k)})\in E_t^{(\text{intent})}\quad \text{for }k=1,\ldots,K_0,\]
  with \textbf{decay weights}
  \(W_t^{(\text{intent})}[v_i^{t},u_{t,i}^{(k)}]=\alpha_k\) where
  \(\alpha_1\ge\cdots\ge \alpha_{K_0}>0\) (e.g.,
  \(\alpha_k\in\{1.0,0.8,0.6,0.4,0.2\}\)).
\end{enumerate}

The learning context is a length-\(H\) \textbf{temporal window}
\[\mathcal{G}_{t-H+1:t}=\big\{G_{\tau}\big\}_{\tau=t-H+1}^{t},\] used to
predict vehicle ETAs at time \(t\).

\section{Results}\label{results}

Run the actual system, evaluate results, and compare to baselines.

\section{Discussion}\label{discussion}

Reflect on the results. Where did your model perform well? Where did it
struggle? Why?

\section{Conclusion}\label{conclusion}

Summarize what was learned and propose next steps.

\section{Acknowledgments}\label{acknowledgments}

This should be a simple paragraph before the References to thank those
individuals and institutions who have supported your work on this
article.

\section*{References}\label{references}
\addcontentsline{toc}{section}{References}

\phantomsection\label{refs}
\begin{CSLReferences}{0}{0}
\bibitem[\citeproctext]{ref-derrowpinion2021googlemaps}
\CSLLeftMargin{{[}1{]} }%
\CSLRightInline{A. Derrow-Pinion \emph{et al.}, {``ETA prediction with
graph neural networks in google maps,''} in \emph{Proceedings of the
27th ACM SIGKDD conference on knowledge discovery \& data mining}, 2021,
pp. e.g., 4034--4042. doi:
\href{https://doi.org/10.1145/3459637.3481916}{10.1145/3459637.3481916}.}

\bibitem[\citeproctext]{ref-hoseinzadeh2020waze}
\CSLLeftMargin{{[}2{]} }%
\CSLRightInline{N. Hoseinzadeh, Y. Liu, L. D. Han, C. Brakewood, and A.
Mohammadnazar, {``Quality of location-based crowdsourced speed data on
surface streets: A case study of waze and bluetooth speed data in
sevierville, TN,''} \emph{Computers, Environment and Urban Systems},
vol. 83, p. 101518, 2020, doi:
\href{https://doi.org/10.1016/j.compenvurbsys.2020.101518}{10.1016/j.compenvurbsys.2020.101518}.}

\bibitem[\citeproctext]{ref-amin-naseri2018waze}
\CSLLeftMargin{{[}3{]} }%
\CSLRightInline{M. Amin-Naseri, P. Chakraborty, A. Sharma, S. B.
Gilbert, and M. Hong, {``Evaluating the reliability, coverage, and added
value of crowdsourced traffic incident reports from waze,''}
\emph{Transportation Research Record}, vol. 2672, no. 43, pp. 34--43,
2018, doi:
\href{https://doi.org/10.1177/0361198118790619}{10.1177/0361198118790619}.}

\bibitem[\citeproctext]{ref-dijkstra1959}
\CSLLeftMargin{{[}4{]} }%
\CSLRightInline{E. W. Dijkstra, {``A note on two problems in connexion
with graphs,''} \emph{Numerische Mathematik}, vol. 1, no. 1, pp.
269--271, 1959, doi:
\href{https://doi.org/10.1007/BF01386390}{10.1007/BF01386390}.}

\bibitem[\citeproctext]{ref-chen2016xgboost}
\CSLLeftMargin{{[}5{]} }%
\CSLRightInline{T. Chen and C. Guestrin, {``XGBoost: A scalable tree
boosting system,''} in \emph{Proceedings of the 22nd ACM SIGKDD
international conference on knowledge discovery and data mining}, ACM,
2016, pp. 785--794. doi:
\href{https://doi.org/10.1145/2939672.2939785}{10.1145/2939672.2939785}.}

\bibitem[\citeproctext]{ref-nyc_tlc}
\CSLLeftMargin{{[}6{]} }%
\CSLRightInline{NYC TLC, {``New york city taxi and limousine commission
trip record data.''}
\url{https://www.nyc.gov/site/tlc/about/tlc-trip-record-data.page},
2013-\/-.}

\bibitem[\citeproctext]{ref-moreira2013porto}
\CSLLeftMargin{{[}7{]} }%
\CSLRightInline{L. Moreira-Matias, J. Gama, M. Ferreira, J.
Mendes-Moreira, and L. Damas, {``Predicting taxi--passenger demand using
streaming data,''} in \emph{2013 IEEE 16th international conference on
intelligent transportation systems (ITSC)}, IEEE, 2013, pp. 140--145.
doi:
\href{https://doi.org/10.1109/ITSC.2013.6728228}{10.1109/ITSC.2013.6728228}.}

\bibitem[\citeproctext]{ref-wang2018deeptte}
\CSLLeftMargin{{[}8{]} }%
\CSLRightInline{J. Wang, Y. Fu, and Z. Zhang, {``When will you arrive?
Estimating travel time based on deep neural networks,''} in
\emph{Proceedings of the AAAI conference on artificial intelligence},
2018, pp. 2500--2507. doi:
\href{https://doi.org/10.1609/aaai.v32i1.11772}{10.1609/aaai.v32i1.11772}.}

\bibitem[\citeproctext]{ref-xu2020tadnm}
\CSLLeftMargin{{[}9{]} }%
\CSLRightInline{W. Xu, Z. Lin, Y. Zhao, T. Zhang, and B. Yang, {``TADNM:
A transportation-mode aware deep neural model for travel time
estimation,''} \emph{Applied Sciences}, vol. 10, no. 21, p. 7599, 2020,
doi: \href{https://doi.org/10.3390/app10217599}{10.3390/app10217599}.}

\bibitem[\citeproctext]{ref-wang2022metatte}
\CSLLeftMargin{{[}10{]} }%
\CSLRightInline{X. Wang, J. Li, and N. J. Yuan, {``MetaTTE: A
meta-learning framework for travel time estimation,''} in
\emph{Proceedings of the 28th ACM SIGKDD conference on knowledge
discovery and data mining}, ACM, 2022, pp. 4080--4088. doi:
\href{https://doi.org/10.1145/3534678.3539066}{10.1145/3534678.3539066}.}

\bibitem[\citeproctext]{ref-abbar2020stad}
\CSLLeftMargin{{[}11{]} }%
\CSLRightInline{S. Abbar, A. Anagnostopoulos, S. Bhagat, P.
Cudre-Mauroux, and A. Kumar, {``STAD: Spatio-temporal adjustment for
improving travel-time estimation,''} in \emph{Proceedings of the web
conference 2020}, ACM, 2020, pp. 2839--2845. doi:
\href{https://doi.org/10.1145/3366423.3380127}{10.1145/3366423.3380127}.}

\bibitem[\citeproctext]{ref-li2018dcrnn}
\CSLLeftMargin{{[}12{]} }%
\CSLRightInline{Y. Li, R. Yu, C. Shahabi, and Y. Liu, {``Diffusion
convolutional recurrent neural network: Data-driven traffic
forecasting,''} in \emph{International conference on learning
representations}, 2018.}

\bibitem[\citeproctext]{ref-yu2018stgcn}
\CSLLeftMargin{{[}13{]} }%
\CSLRightInline{B. Yu, H. Yin, and Z. Zhu, {``Spatio-temporal graph
convolutional networks: A deep learning framework for traffic
forecasting,''} in \emph{Proceedings of the 27th international joint
conference on artificial intelligence}, 2018, pp. 3634--3640. doi:
\href{https://doi.org/10.24963/ijcai.2018/505}{10.24963/ijcai.2018/505}.}

\bibitem[\citeproctext]{ref-wu2019graphwavenet}
\CSLLeftMargin{{[}14{]} }%
\CSLRightInline{Z. Wu, S. Pan, G. Long, J. Jiang, and C. Zhang, {``Graph
WaveNet for deep spatial-temporal graph modeling,''} in
\emph{Proceedings of the 28th international joint conference on
artificial intelligence}, 2019, pp. 1907--1913. doi:
\href{https://doi.org/10.24963/ijcai.2019/264}{10.24963/ijcai.2019/264}.}

\bibitem[\citeproctext]{ref-didi2016}
\CSLLeftMargin{{[}15{]} }%
\CSLRightInline{DiDi Chuxing Research, {``Di-tech challenge 2016.''}
\url{https://outreach.didichuxing.com/research/opendata/en/}, 2016.}

\bibitem[\citeproctext]{ref-zheng2012geolife}
\CSLLeftMargin{{[}16{]} }%
\CSLRightInline{Y. Zheng, L. Zhang, Z. Ma, X. Xie, and W.-Y. Ma,
{``Mining interesting locations and travel sequences from GPS
trajectories,''} in \emph{Proceedings of the 18th international
conference on world wide web}, ACM, 2009, pp. 791--800. doi:
\href{https://doi.org/10.1145/1526709.1526816}{10.1145/1526709.1526816}.}

\bibitem[\citeproctext]{ref-voloch2021}
\CSLLeftMargin{{[}17{]} }%
\CSLRightInline{N. Voloch and N. Voloch-Bloch, {``Finding the fastest
navigation route by real-time future traffic estimations,''} \emph{2021
IEEE International Conference on Microwaves, Communications, Antennas
and Electronic Systems (COMCAS)}, 2021, doi:
\href{https://doi.org/10.1109/COMCAS52884.2021.9628700}{10.1109/COMCAS52884.2021.9628700}.}

\bibitem[\citeproctext]{ref-goldberg2005alt}
\CSLLeftMargin{{[}18{]} }%
\CSLRightInline{A. V. Goldberg and C. Harrelson, {``Computing the
shortest path: A* search meets graph theory,''} in \emph{Proceedings of
the sixteenth annual ACM-SIAM symposium on discrete algorithms (SODA)},
2005, pp. 156--165.}

\bibitem[\citeproctext]{ref-geisberger2008ch}
\CSLLeftMargin{{[}19{]} }%
\CSLRightInline{R. Geisberger, P. Sanders, D. Schultes, and D. Delling,
{``Contraction hierarchies: Faster and simpler hierarchical routing in
road networks,''} in \emph{International workshop on experimental and
efficient algorithms (WEA)}, Springer, 2008, pp. 319--333.}

\bibitem[\citeproctext]{ref-abraham2010highwaydimension}
\CSLLeftMargin{{[}20{]} }%
\CSLRightInline{I. Abraham, A. Fiat, A. V. Goldberg, and R. F. Werneck,
{``Highway dimension, shortest paths, and provably efficient
algorithms,''} in \emph{Proceedings of the 21st annual ACM-SIAM
symposium on discrete algorithms (SODA)}, 2010, pp. 782--793. doi:
\href{https://doi.org/10.1137/1.9781611973075.64}{10.1137/1.9781611973075.64}.}

\bibitem[\citeproctext]{ref-delling2011crp}
\CSLLeftMargin{{[}21{]} }%
\CSLRightInline{D. Delling, A. V. Goldberg, T. Pajor, and R. F. Werneck,
{``Customizable route planning,''} in \emph{International symposium on
experimental algorithms (SEA)}, Springer, 2011, pp. 376--387.}

\bibitem[\citeproctext]{ref-yen1971ksp}
\CSLLeftMargin{{[}22{]} }%
\CSLRightInline{J. Y. Yen, {``Finding the k shortest loopless paths in a
network,''} in \emph{Management science}, 1971, pp. 712--716. doi:
\href{https://doi.org/10.1287/mnsc.17.11.712}{10.1287/mnsc.17.11.712}.}

\end{CSLReferences}




\end{document}
