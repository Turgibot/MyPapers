\section{Related Work}

\subsection{Classical pathfinding and routing}
On static graphs $G=(V,E)$ with non-negative edge costs, shortest paths are classically computed via Dijkstra’s algorithm~\cite{dijkstra1959}. Alternatives such as Bellman–Ford~\cite{bellman1958routing} and bidirectional search~\cite{pohl1971bi} trade efficiency for generality. More advanced techniques such as A*/ALT landmarks~\cite{goldberg2005}, Contraction Hierarchies~\cite{geisberger2008}, and Customizable Route Planning~\cite{delling2011} scale to web applications but rely on static edge-time models. As a result, they fail to capture evolving congestion or vehicle interactions. Our approach differs fundamentally by learning ETA directly from dynamic traffic graphs rather than static cost assumptions.

\subsection{Learning-based ETA and traffic forecasting}
The proliferation of urban mobility datasets motivated machine learning methods for ETA. Gradient boosting models such as XGBoost~\cite{chen2016xgboost} leverage handcrafted features (trip distance, time-of-day) and perform well on aggregated taxi records (e.g., NYC, Porto~\cite{nyc_tlc,moreira2013porto}), but they cannot capture fine-grained spatio-temporal interactions.

Trajectory- and trip-based neural models address this gap. DeepTTE~\cite{deepTTE2018} learns ETA from raw GPS traces and reports results on city-scale taxi/ride-hailing datasets (e.g., NYC TLC~\cite{nyc_tlc}, Porto Taxi~\cite{moreira2013porto}, DiDi 2016~\cite{didi2016}) with per-trip horizons typically within tens of minutes; MetaTTE~\cite{wang2022metatte} and related methods exploit cross-domain adaptation; STAD~\cite{abbar2020stad} adjusts routing outputs with spatio-temporal corrections. Production systems such as STANN~\cite{stann2021} evaluate on large ride-hailing ETA datasets with short-to-medium trip durations (e.g., sub-30-minute trips). Complementary to these, the Smart Simulative Route framework~\cite{SmartSimulativeRoute2025} anticipates future congestion via heuristic simulation to improve ETA, but it remains simulation-driven and does not leverage graph-structured, data-driven temporal learning. Our work instead integrates explicit route intent and dynamic graph learning in a unified model.

\subsection{Spatio-temporal graph neural networks}
Graph neural networks advanced traffic forecasting by modeling sensor networks as spatio-temporal graphs. DCRNN~\cite{dcrnn2018} combined diffusion graph convolutions with recurrent units and became a common backbone for traffic speed/flow prediction on sensor benchmarks (METR-LA, PEMS-BAY; 5–60 min horizons in speed units). Notably, DCRNN-style encoders have also been adapted for ETA prediction in production settings, e.g., within DuETA~\cite{dueta2023}. In contrast to sensor-grid forecasting, we target per-vehicle ETA on dynamic multi-relational graphs with explicit route intent, while ConSTGAT~\cite{constgat2020} targets sensor-grid speed forecasting on METR-LA/PEMS-BAY at fixed horizons (not per-trip ETA).

Importantly, reported MAEs across these works are not directly comparable without dataset and time-horizon context (city, data source, and maximum trip duration). We therefore reference methods together with their evaluation scope rather than quoting unscoped headline numbers.

\subsection{State-of-the-art production models}
Most recently, DuETA~\cite{dueta2023} introduced duration-aware ETA modeling at Baidu Maps, categorizing trips into short (0--3 km), medium (3--10 km), and long ($>$10 km). DuETA achieved MAEs of 27s, 46s, and 98s respectively across these categories, significantly improving upon prior baselines. Unlike DuETA, which relies on trip segmentation, our method unifies all trip types under a single dynamic spatio-temporal framework. By incorporating route-left path features directly into the graph, our model can generalize across short and long trips without predefined bins. Moreover, we learn temporal context from stable road edges across history and fuse it at the prediction step, complementing DuETA's duration-aware design with explicit route intent and road-based temporal aggregation.

\subsection{Summary}
Prior work highlights the value of sequential modeling (DeepTTE), spatio-temporal attention (STANN, ConSTGAT), graph diffusion (DCRNN), and duration-aware embeddings (DuETA). However, none fully integrate explicit pre-planned routes with vehicle–junction dynamics within a unified dynamic graph. Our contribution is to combine explicit route intent, road-based temporal aggregation, and a sparse MoE head in a single spatio-temporal GNN for ETA.
