\section{Conclusion}

This work introduced a dynamic spatio-temporal graph neural network for ETA prediction that unifies static infrastructure, vehicle-level dynamics, and explicit route intent within a Mixture-of-Experts framework. A central innovation of our approach is the formulation of the traffic environment as a dynamic graph in which both junctions and vehicles are nodes, connected by persistent road edges and evolving interaction edges. This representation enables the model to leverage the stability of the road network while simultaneously adapting to transient traffic patterns.  

By systematically evaluating ablation variants, we demonstrated that each component contributes to predictive accuracy: graph structure captures spatial constraints, route features provide the largest single boost, and temporal aggregation refines predictions by modeling evolving congestion. The resulting temporal route-aware model reduced validation MAE from 260.6\,s in the average baseline to 46.2\,s, an 82.3\% improvement, with balanced accuracy across short, medium, and long trips. Compared with improvements typically reported in the range of 40–50\% by state-of-the-art baselines, these results highlight the benefit of representing the traffic system as a dynamic vehicle–junction graph.  

Future work will focus on scaling to real-world datasets, addressing error accumulation in long trips, and ensuring robustness under unexpected disruptions, thereby advancing the practical deployment of intent-aware dynamic graph models in intelligent transportation systems.
