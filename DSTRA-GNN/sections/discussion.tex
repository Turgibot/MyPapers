\section{Discussion}

The results highlight three clear trends. First, static road graphs already provide a strong inductive bias, with the \texttt{base\_graph} achieving a two-thirds reduction in error relative to the average baseline. This demonstrates the value of graph-structured representations in capturing spatial constraints.

Second, route intent emerges as the most decisive feature. The \texttt{route\_aware\_graph} achieves 58.4\,s MAE overall, more than 40\% lower than either the \texttt{base\_graph} or \texttt{dynamic\_graph}. Notably, short trips drop from 39--43\,s error to just 24\,s, indicating that route features resolve much of the ambiguity about vehicle destination and path choice.

Third, temporal context provides an additional layer of refinement. The \texttt{temporal\_route\_aware} variant achieves 46.2\,s MAE and 17.9\% MAPE, corresponding to an 82.3\% improvement over the average baseline and outperforming all other variants across every trip-length category. The temporal aggregation of $T{-}1$ snapshots enables the network to recognize evolving congestion patterns, improving robustness for medium and long trips where static and route-only models tend to accumulate error.

These gains are substantially higher than those reported by state-of-the-art methods on real-world datasets. For example, DuETA reports MAE reductions of 40--50\% relative to average baselines across Beijing, Shanghai, and Tianjin~\cite{dueta2023}, while earlier models such as DCRNN, DeepTTE, and ConSTGAT report similar ranges~\cite{dcrnn2018,deepTTE2018,constgat2020}. In contrast, our model delivers over 80\% improvement, underscoring the benefit of combining explicit route intent with dynamic spatio-temporal graph modeling.

Together, these findings validate the dataset design, which provides per-vehicle route annotations, dynamic interaction edges, and temporally aligned graph snapshots. Each of these components is directly reflected in the ablation gains. At the same time, long trips remain the most challenging case (76.9\,s MAE even in the best model), indicating that compounding local errors across extended horizons remains an open problem.

A further contribution of this work lies in the dataset formulation itself. By explicitly representing both junctions and vehicles as nodes, and layering dynamic edges on top of static road topology, the traffic network is captured in a richer and more flexible way than traditional road-segment–based graphs. This representation provides the inductive bias needed to integrate infrastructure constraints with real-time vehicle interactions, and the ablation results demonstrate that it is precisely this combination—static roads, dynamic edges, and route-aware features—that yields the largest performance gains.

The scale and temporal coverage of the dataset further reinforce these results. With over 80,000 graph snapshots, 200,000+ vehicles, and approximately 927,000 distinct routes across a continuous four-week horizon, the evaluation provides dense temporal coverage and rich route diversity. This extended duration ensures that models are exposed to recurring traffic cycles such as rush hours, weekday–weekend variations, and long-term congestion patterns, while the large number of routes captures heterogeneous travel intents and path choices. Compared to widely used benchmarks such as METR-LA (342 loop sensors, aggregated speeds over a few months) or trajectory datasets like Q-Traffic and DiDi, which provide trip-level GPS traces without explicit route intent or graph structure, our dataset offers far richer temporal, spatial, and behavioral signals. Importantly, this level of detail reflects information readily available in modern navigation systems through GPS traces, map matching, and planned-route logging, making the evaluation both realistic and robust while reducing the risk of overfitting to short-term artifacts.
